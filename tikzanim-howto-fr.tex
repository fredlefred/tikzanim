\documentclass[a4paper,12pt]{article}

\usepackage[T1]{fontenc}
\usepackage{amsmath}
\usepackage{multicol}
\usepackage{enumitem}
\usepackage[margin=1cm,includefoot]{geometry}
\usepackage{tikzanim}
\usepackage[francais]{babel}
\usetikzlibrary{calc,intersections,tikzmark,decorations.pathreplacing}

\usepackage{hyperref}

\usepackage{listings}
\lstset{
	basicstyle=\ttfamily\footnotesize,
	columns=fullflexible,
	language={[LaTeX]TeX},
	tabsize=2,
	breaklines,
	xleftmargin=2em,
	backgroundcolor=\color{black!10},
	numbers=left,
	numberstyle=\color{black!50},
	commentstyle=\color{gray},
	keywordstyle=\color{teal!75!black},
	morekeywords={draw,coordinate,useasboundingbox},
	identifierstyle=\color{blue!75!black},
	emph={tikzanim,step,allframes,framepos},
	emphstyle=\color{purple},
	commentstyle=\color{green!75!black},
}

\tikzset{construct line/.style={densely dashed,black!25}}

\def\keywords{make easily animation with animate package and tikz}
\hypersetup{
  pdftitle={Le Package tikzanim},
  pdfsubject={Documentation},
  pdfauthor={Frédéric Bonnaud},
  pdfkeywords={\keywords},
    pdfborder = {0 0 0},
    colorlinks,
    citecolor=red,
    filecolor=Darkgreen,
    linkcolor=blue,
    urlcolor=cyan!50!black!90
}
\author{Frédéric Bonnaud}%
\title{\TikzAnimate\\ Créer une animation avec \Tikz}
\date{15 juin 2020}

\newcommand\Tikz{{\ttfamily tikz}}
\newcommand\TikzAnimate{{\ttfamily tikzanim}}
\newcommand\Animate{{\ttfamily animate}}

\setlength{\parindent}{0mm}
\begin{document}

\maketitle

\tableofcontents

\section{Introduction}

\TikzAnimate\ permet de réaliser simplement des animations en \LaTeX à l'aide de \Tikz\ et \Animate\ en décrivant les différentes étapes de l'animation par un code \Tikz\ standard.

Une animation est constituée d'un certaine nombre d'étapes. Chaque étape dessine des transparents qui sont combinés les uns aux autres pour créer les images de l'étape d'animation qu'elle constitue.


\section{Les macros offertes par \TikzAnimate}

\subsection{\ttfamily\textbackslash tikzanim}

	Cette macro installe tout ce qu'il faut pour mettre en place l'animation : un environnement {\ttfamily animateinline} (pour créer l'animation), la macro {\ttfamily\textbackslash multiframe} (pour créer les différents images) et un environnement {\ttfamily\textbackslash tikzpicture} (pour dessiner les différentes images).
	
	\textbf{Syntaxe~:} 
	
	\quad {\ttfamily \textbackslash tikzanim} {\ttfamily <overlay>} {\ttfamily [options animateinline]} {\ttfamily \{rafraîchissement\}}
	
	{\ttfamily [options tikz]} {\ttfamily \{initialisation\}} {\ttfamily \{étapes\}}
	
	\begin{itemize}
		\item {\ttfamily <overlay>} : le numero de l'{\ttfamily overlay} sur lequel sera affiché l'animation. Avant celui-ci, {\ttfamily\textbackslash tikzanim} affiche la première image, après celui-ci il affiche la dernière image. 
		
		\emph{Remarque : il est préférable que l'{\ttfamily overlay} se limite à un seul transparent, sinon l'animation sera générée de multiple
		fois.}
		\item {\ttfamily [options animateinline]} : les options qui seront passées à l'environnement {\ttfamily animateinline}.
		
		Valeur par défaut : {\ttfamily poster=last,controls}
		\item {\ttfamily \{rafraîchissement\}} : le nombre d'images par seconde au début de l'animation. Il peut être modifié à chaque étape.
		\item {\ttfamily [options tikz]} : les options qui seront passées à l'environnement {\ttfamily tikzpicture}.
		
		Valeur par défaut : pas d'option
		\item {\ttfamily \{initialisation\}} : le code d'initialisation. Ce code sera appelé pour tous les transparents. À minima, ce code devrait contenir un appel à {\ttfamily\textbackslash useasboundingbox} pour fixer la même taille à tous les transparents.
		\item {\ttfamily \{étapes\}} : la suite des différentes étapes de l'animation. 
	\end{itemize}
	
\subsection{\ttfamily\textbackslash step}

	Cette macro définit une nouvelle étape de l'animation. \Animate\ définit une pile de transparents. Chaque image de l'animation est générée par l'affichage 
	de toute la pile de transparents à chaque instant. Le rôle de {\ttfamily\textbackslash step} est de gérer la création (avec \Tikz) et la gestion cette pile
	(en générant un fichier timeline) pour créer les différents transparents qui permettrons créer les images de l'animation.
		
	Au début de l'animation, la pile est vide. Un appel à {\ttfamily\textbackslash step} va :
	
	\begin{enumerate}[label=(\arabic*)]
		\item Générer le premier transparent cette étape et le déposer sur la pile pour une durée de une image, puis générer la première image.
		\item \tikzmark{boucle}Enlever le dernier transparent la pile.
		\item Générer le transparent suivant et le déposer sur la pile pour une durée de un image, puis générer l'image suivante.
		\item \tikzmark{test}Si ce n'est pas la fin de l'étape, on boucle vers (2).
		\item Si c'est la fin de l'étape, on laisse le dernier transparent pour une durée définie.
		
	\end{enumerate}
	\begin{tikzpicture}[remember picture,overlay]
		\coordinate(T)at(-10mm,1mm);
		\draw[-latex,thick] ($(pic cs:test)+(T)$) to[bend left] ($(pic cs:boucle)+(T)$) ; 
	\end{tikzpicture}
	
	\textbf{Syntaxe :}
	
	\quad {\ttfamily\textbackslash step[*] [rafraîchissement] \{images\} [durée] \{dessin\} }
	
	\begin{itemize}
		\item {\ttfamily *} : la version étoilée, vide la pile de transparents au début de l'étape.
		\item {\ttfamily[rafraîchissement]} : définit le nouveau nombre d'images par seconde.
		
		Valeur par défaut : 0, c'est à dire le dernier nombre d'images par seconde définit par {\ttfamily\textbackslash tikzanim} ou {\ttfamily\textbackslash step}.
		\item {\ttfamily\{images\}} : définit le nombre de images que doit générer {\ttfamily\textbackslash step}. Si ce nombre est 0. Le transparent est déposé sur la pile pour la durée
		définie, mais aucune image n'est créée pour celui-ci. Cela permet de dessiner dans des étapes différentes des objets qui auront des durées d'affichage différentes.
		
		\item {\ttfamily[durée]} : définit le temps pendant lequel le dernier transparent doit durer (en nombre d'images).
		
		Valeur par défaut : 0, c'est à dire jusqu'à la fin de l'animation.

		\item {\ttfamily\{dessin\}} : définit le code \Tikz\ chargé de dessiner les différents transparents. Ce code peut utiliser tout ce qui se trouve dans
		les codes d'initialisation des étapes précédentes, ainsi que les macros {\ttfamily\textbackslash framepos} et {\ttfamily\textbackslash iframe} pour que 
		ce qui est dessiné dépende de l'image en train d'être dessinée.
		
		\begin{itemize}
			\item La macro {\ttfamily\textbackslash framepos} vaut 0 au  début de l'étape et varie de façon linéaire jusqu'à 1 à la fin de l'étape. Cette macro est gérée par {\ttfamily\textbackslash step}.
			\item La macro {\ttfamily\textbackslash iframe} est égale au numéro du transparent en cours de génération. Cette macro est gérée par {\ttfamily\textbackslash multiframe} du package \Animate.
		\end{itemize}
		
	\end{itemize}
	
	\begin{multicols}{2}
	{Le code suivant~:}
	\begin{lstlisting}
% \usepackage{tikzanim}
\tikzanim{10}{
	% cette étape se prolongera sur 15 images
	% c'est à dire la moitié de l'étape suivante
	\step{30}[15]{
		% il faut initialiser A et B ici, car
		% la `boundingbox' doit être utilisée
		% à l'étape suivante
		\coordinate(A)at(0,0) ;
		\coordinate(B)at(5,5) ;
		\useasboundingbox(A)rectangle(B) ;
	}{
		\draw(A)--($(A)!\framepos!(B)$) ;
	}
	% la seconde étape sera 2 fois plus rapide
	\step[20]{30}{
		% pas d'initialisation, C et D ne
		% sont pas utilisés par la suite
	}{
		\coordinate(C)at(0,5) ;
		\coordinate(D)at(5,0) ;
		\draw(C)--($(C)!\framepos!(D)$) ;
	}
}
	\end{lstlisting}
	
	{Crée l'animation ci-contre~:}
	
	\bigskip
	
	\centering
	% \usepackage{tikzanim}
	\tikzanim{10}{
		\useasboundingbox(-0.1,0.1)rectangle(5.1,2.6) ;
	}{
		\step{30}[15]{
			\coordinate(A)at(0,0) ;
			\coordinate(B)at(5,2.5) ;
			\draw(A)--($(A)!\framepos!(B)$) ;
		}
		\step[20]{30}{
			% pas d'initialisation, C et D ne
			% sont pas utilisés par la suite
			\coordinate(C)at(0,2.5) ;
			\coordinate(D)at(5,0) ;
			\draw(C)--($(C)!\framepos!(D)$) ;
		}
	}
	
	\end{multicols}
	
	\textbf{Schéma de la pile de transparents~:}
	
	\medskip
	
\newcounter{dy} 
\newcommand{\transparentliste}[2]{
	\stepcounter{dy}	
	\foreach\f in {#1}{
		\pgfmathsetmacro{\c}{\f*5}
		\path[draw=black,fill=teal!\c!purple!75!white,fill opacity=0.75] (\f*0.1,-\thedy*0.8)node(a){} -- ++(0,0.6) -- ++(0.3,0.3)node(b){} -- ++(0,-0.6) -- cycle ;
		\path(a)--node[midway,font=\tiny,shape=rectangle]{\f} (b) ;
	}
	\node[anchor=west,shape=rectangle] at(2.5,0.4-\thedy*0.8) {#2#1} ;
}

\newcommand{\acc}[4][0pt]{
		\draw[decorate,decoration={brace,amplitude=10pt},thick,xshift=#1] (12,-0.8*#2-0.05) --node[right,xshift=10pt,text width=7cm,font=\ttfamily,shape=rectangle]{#4} (12,-0.8*#3+0.05);
}

\newcommand{\entoure}[4][]{
	\filldraw[red,thick,densely dashed,fill opacity=0.333,#1] (#2*0.1-0.05,-#4*0.8-0.125) -- ++(0,{(#4-#3+1)*0.8-0.05}) -- ++(0.4,0.4) -- ++ (0,{-(#4-#3+1)*0.8+0.05}) -- cycle ;
}

\newcommand{\limite}[2][teal!50,thick,densely dotted]{
	\draw[#1](-0.2,-0.8*#2) -- (11.9,-0.8*#2) ;
}


\begin{tikzpicture}[shape=coordinate]

	\entoure{0}{1}{18}

	%\def\label{Liste de transparents : }
	%\transparentliste{0,...,20}{\label}
	\def\label{Image \thedy, constituée des transparents : }
	
	
	\transparentliste{0,1}{\label}
	\transparentliste{0,2}{\label}
	\transparentliste{0,3}{\label}
	\transparentliste{0,4}{\label}
	\transparentliste{0,5}{\label}
	
	\entoure{6}{6}{9}
	
	\transparentliste{0,6}{\label}
	
	\acc{0}{6}{ 
		\textbackslash step\{0\}{\color{red}[18]}\{...\}\\ 
		{\rmfamily ajoute 1 transparent qui dure {\color{red}18 images}, sans ajouter d'image}\\
		\textbackslash step\{6\}{\color{red}[4]}\{...\}\\
		{\rmfamily ajoute 6 transparents et 6 images, le dernier dure \color{red}4 images}
	}
	
	\limite{6}
	
	\transparentliste{0,6,7}{\label}
	\transparentliste{0,6,8}{\label}
	\transparentliste{0,6,9}{\label}
	\transparentliste{0,10}{\label}
	\transparentliste{0,11}{\label}

	\acc{6}{11}{ 
		\textbackslash step\{5\}\{...\} \\
		{\rmfamily ajoute 5 transparents et 5 images}
	}
	
	\limite{11}
	
	\transparentliste{0,11,12}{\label}
	\transparentliste{0,11,13}{\label}
	\transparentliste{0,11,14}{\label}
	\transparentliste{0,11,15}{\label}
	
	\entoure{16}{16}{18}
	\transparentliste{0,11,16}{\label}
	
	\acc{11}{16}{
		\textbackslash step\{5\}{\color{red}[3]}\{...\} \\
		{\rmfamily ajoute 5 transparents et 5 images, le dernier dure \color{red}3 images}
	}
	
	\limite{16}
	
	\transparentliste{0,11,16,17}{\label}
	\transparentliste{0,11,16,18}{\label}
	\transparentliste{11,19}{\label}
	\transparentliste{11,20}{\label}
	
	\acc{16}{20}{
		\textbackslash step\{4\}\{...\} \\
		{\rmfamily ajoute 4 transparents et 4 images}
	}

\end{tikzpicture}
	

\subsection{\ttfamily\textbackslash allframes}

Cette macro permet de dessiner sur tous les transparents.

\textbf{Syntaxe~:}

\quad {\ttfamily\textbackslash allframes \{dessin\}}

	\begin{itemize}
		\item {\ttfamily\{dessin\}} : définit le code \Tikz\ chargé de dessiner sur tous les transparents. Ce code peut utiliser tout ce qui se trouve dans
		les codes d'initialisation des étapes précédentes, ainsi que {\ttfamily\textbackslash framepos} et {\ttfamily\textbackslash iframe}.
	\end{itemize}
	
\section{\ttfamily timeline}

\Animate\ utilise un fichier {\ttfamily timeline} pour gérer l'affichage des transparents. \TikzAnimate\ crée un fichier {\ttfamily timeline} par animation. Ils sont nommés : {\ttfamily \textbackslash jobname.tzc\#.tln}. Ils peuvent être utiliser à des fins de débuggage de l'animation.
	
\section{Un exemple complet}

\begin{lstlisting}[name=exemplecomplet]
\gdef\aLen{5}
\gdef\bLen{2}
\pgfmathsetmacro{\cLen}{sqrt(\aLen^2+\bLen^2)}
\tikzset{bleu/.style={fill=blue!30,fill opacity=0.5},rouge/.style={fill=red!30,fill opacity=0.5}} 
\end{lstlisting}

Quelques initialisations.


\begin{lstlisting}[name=exemplecomplet]

\tikzanim[poster=0,controls]{1}{
	% initialisation 
	\useasboundingbox(-0.1,-\bLen-0.1)rectangle(\aLen+\bLen+0.1,\aLen+0.1) ;
}
\end{lstlisting}
On crée une animation à une image par seconde (au départ).

Puis, on crée les points servants à la construction, en même temps qu'un permier transparent qui ne sera affiché qu'à la prochaine image.
\begin{lstlisting}[name=exemplecomplet]
{
	\step{0}[1]{		
		\coordinate(A)at(0,0) ;
		\coordinate(B)at(\aLen,0) ;
		\coordinate(C)at(\aLen,\aLen) ;
		\coordinate(D)at(0,\aLen) ;

		\coordinate(E)at(\aLen+\bLen,0) ;
		\coordinate(F)at(\aLen+\bLen,\bLen) ;
		\coordinate(G)at(\aLen,\bLen) ;

		\path[name path=EC](E)--(C) ;
		\path[name path=FG](F)--(G) ;
		\path[name intersections={of=EC and FG,by=H}] ;

		\coordinate(I)at(0,\aLen-\bLen) ;
		\coordinate(J)at($(A)+(H)-(G)$);
		% premier morceau
		\fill[bleu] (C) -- (D) -- (I) -- cycle ;
	}
\end{lstlisting}

On crée un nouveau transparent qui ne sera affiché qu'avec la prochaine image jusqu'à jusqu'à la fin de la $4^\text{ème}$ étape après elle..

\begin{lstlisting}[name=exemplecomplet]
	\step{0}[4.0]{
		% deuxième morceau
		\fill[bleu] (I) -- (A) -- (J) -- cycle ;
	}
\end{lstlisting}

On crée encore un nouveau transparent qui ne sera affiché qu'avec la prochaine image jusqu'à la fin de la $5^\text{ème}$ étape après elle. .

\begin{lstlisting}[name=exemplecomplet]
	\step{0}[5.0]{
		% troisième morceau
		\fill[rouge] (E) -- (F) -- (H) -- cycle ;
	}
\end{lstlisting}

On crée un seul transparent, qui contient ce qui va rester affiché jusqu'à la fin (et qui donc ne bougera pas) et les 3 transparents précédants qui disparaîtront lorsque le mouvement de ce qu'ils 
contiennent commencera.

\begin{lstlisting}[name=exemplecomplet]
	\step{1}{
		% figure de départ sans les morceaux
		\draw (A) -- (B) -- (C) -- (D) --cycle ;
		\draw (B) -- (E) -- (F) -- (G) --cycle ;
		
		\fill[bleu] (I) -- (J) -- (B) -- (C) -- cycle ;
		\draw (J) -- (B) -- (C) ; \draw[densely dotted] (C) -- (I) -- (J) ;
			
		\fill[rouge] (B) -- (E) -- (H) -- (G) -- cycle ;
		\draw (H) -- (G) -- (B) -- (E) ; \draw[densely dotted] (E) -- (H) ;
	}
\end{lstlisting}
Le mouvement commence, on fixe le nombre d'image par seconde à 5, et on génére 10 images, cette étape va donc prendre 2 secondes. Comme sa durée est de 1 image, cela signifie 
que le dernier transparent ne sera pas affiché sur les images qui suivent cette étape.
\begin{lstlisting}[name=exemplecomplet]
	% déplacement du premier triangle
	\step[5]{10}[1]{
		\coordinate(M)at($(C)!\framepos!(E)$);% Point à \framepos sur [CE]
		\coordinate(CM)at($(M)-(C)$) ;
		\coordinate(ICM)at($(I)+(CM)$);
		\coordinate(CCM)at($(C)+(CM)$);
		\coordinate(DCM)at($(D)+(CM)$);
		% on affiche une flèche en plus du triangle pour bien voir d'où vient le triangle.
		\coordinate(S)at(barycentric cs:I=1,C=1,D=1);
		\coordinate(S')at(barycentric cs:ICM=1,CCM=1,DCM=1);
		\fill[bleu] (ICM) -- (CCM) -- (DCM) -- cycle ;
		\draw[densely dotted] (ICM) -- (CCM) ;
		\draw (CCM) -- (DCM) -- (ICM) ;
		\draw[-latex](S)--(S') ;
	}
\end{lstlisting}

Comme dernier transparent ne dure qu'une image, le triangle final ne reste pas dessiné. On le dessine à nouveau.

\begin{lstlisting}[name=exemplecomplet]
	% position finale
	\step{1}{
		\fill[bleu] (ICM) -- (CCM) -- (DCM) -- cycle ;
		\draw[densely dotted] (ICM) -- (CCM) ;
		\draw (CCM) -- (DCM) -- (ICM) ;
	}
\end{lstlisting}

On procède de même pour le deuxième triangle.

\begin{lstlisting}[name=exemplecomplet]
	% déplacement du deuxième triangle
	\step{10}[1]{
		\coordinate(N)at($(I)!\framepos!(C)$);% Point à \framepos sur [IC]
		\coordinate(IN)at($(N)-(I)$) ;
		\coordinate(IIN)at($(I)+(IN)$);
		\coordinate(JIN)at($(J)+(IN)$);
		\coordinate(AIN)at($(A)+(IN)$);
		\coordinate(T)at(barycentric cs:I=1,J=1,A=1);
		\coordinate(T')at(barycentric cs:IIN=1,JIN=1,AIN=1);
		\fill[bleu] (IIN) -- (JIN) -- (AIN) -- cycle ;
		\draw[densely dotted] (IIN) -- (JIN) ;
		\draw (JIN) -- (AIN) -- (IIN) ;
		\draw[-latex](T)--(T') ;
	}
	% position finale
	\step{1}{
		\fill[bleu] (IIN) -- (JIN) -- (AIN) -- cycle ;
		\draw[densely dotted] (IIN) -- (JIN) ;
		\draw (JIN) -- (AIN) -- (IIN) ;
	}
\end{lstlisting}

On procède de même pour le troisième triangle.

\begin{lstlisting}[name=exemplecomplet]
	% déplacement du troisième triangle
	\step{10}[1]{
		\coordinate(P)at($(H)!\framepos!(J)$);% Point à \framepos sur [HJ]
		\coordinate(HP)at($(P)-(H)$) ;
		\coordinate(EHP)at($(E)+(HP)$);
		\coordinate(FHP)at($(F)+(HP)$);
		\coordinate(HHP)at($(H)+(HP)$);
		\coordinate(U)at(barycentric cs:E=1,F=1,H=1);
		\coordinate(U')at(barycentric cs:EHP=1,FHP=1,HHP=1);
		\fill[rouge] (EHP) -- (FHP) -- (HHP) -- cycle ;
		\draw[densely dotted] (EHP) -- (HHP) ;
		\draw (EHP) -- (FHP) -- (HHP) ;
		\draw[-latex](U)--(U') ;
	}
	% position finale
	\step{1}{
		\fill[rouge] (EHP) -- (FHP) -- (HHP) -- cycle ;
		\draw[densely dotted] (EHP) -- (HHP) ;
		\draw (EHP) -- (FHP) -- (HHP) ;
	}
}
\end{lstlisting}

\textbf{L'animation :}

Une démonstration visuelle du théorème de Pythagore.

\begin{center}
\gdef\aLen{5}
\gdef\bLen{2}
\pgfmathsetmacro{\cLen}{sqrt(\aLen^2+\bLen^2)}
\tikzset{bleu/.style={fill=blue!30,fill opacity=0.5},rouge/.style={fill=red!30,fill opacity=0.5}} 

\tikzanim[poster=0,controls]{5}[>=latex,inner sep=2pt]{
	\useasboundingbox(-0.75,-\bLen-0.75)rectangle(\aLen+\bLen+0.75,\aLen+0.75) ;
}{
	\step[1]{0}[1]{
		
		\coordinate(A)at(0,0) ;
		\coordinate(B)at(\aLen,0) ;
		\coordinate(C)at(\aLen,\aLen) ;
		\coordinate(D)at(0,\aLen) ;
		
		\coordinate(E)at(\aLen+\bLen,0) ;
		\coordinate(F)at(\aLen+\bLen,\bLen) ;
		\coordinate(G)at(\aLen,\bLen) ;

		\path[name path=EC](E)--(C) ;
		\path[name path=FG](F)--(G) ;
		\path[name intersections={of=EC and FG,by=H}] ;

		\coordinate(I)at(0,\aLen-\bLen) ;
		\coordinate(J)at($(A)+(H)-(G)$);
		\fill[bleu] (C) -- (D) -- (I) -- cycle ;
		
		\path(A) --node{$a^2$} (C) ;
		\path(E) --node{$b^2$} (G) ;
		
	}
	\step{0}[4.0]{
		\fill[bleu] (I) -- (A) -- (J) -- cycle ;
	}
	\step{0}[5.0]{
		\fill[rouge] (E) -- (F) -- (H) -- cycle ;
	}
	\step{1}{
		\draw[blue,line width=2pt] (A) -- (B) -- (C) -- (D) --cycle ;
		\draw[red,line width=2pt] (B) -- (E) -- (F) -- (G) --cycle ;
		
		\fill[bleu] (I) -- (J) -- (B) -- (C) -- cycle ;
		\draw (J) -- (B) -- (C) ; \draw[densely dotted] (C) -- (I) -- (J) ;
			
		\fill[rouge] (B) -- (E) -- (H) -- (G) -- cycle ;
		\draw (H) -- (G) -- (B) -- (E) ; \draw[densely dotted] (E) -- (H) ;
		
		\draw[<->]($(I)-(0.3,0)$)--node[fill=white]{$b$} ($(D)-(0.3,0)$) ;
		\draw[<->]($(D)+(0,0.3)$)--node[fill=white]{$a$} ($(C)+(0,0.3)$) ;
		\draw[<->]($(E)+(0.3,0)$)--node[fill=white]{$b$} ($(F)+(0.3,0)$) ;
		
	}

	\step[5]{10}[1]{
		\coordinate(M)at($(C)!\framepos!(E)$);% Point à \framepos sur [CE]
		\coordinate(CM)at($(M)-(C)$) ;
		\coordinate(ICM)at($(I)+(CM)$);
		\coordinate(CCM)at($(C)+(CM)$);
		\coordinate(DCM)at($(D)+(CM)$);
		\coordinate(S)at(barycentric cs:I=1,C=1,D=1);
		\coordinate(S')at(barycentric cs:ICM=1,CCM=1,DCM=1);
		\fill[bleu] (ICM) -- (CCM) -- (DCM) -- cycle ;
		\draw[densely dotted] (ICM) -- (CCM) ;
		\draw (CCM) -- (DCM) -- (ICM) ;
		\draw[->](S)--(S') ;
	}
	
	\step{1}{
		\fill[bleu] (ICM) -- (CCM) -- (DCM) -- cycle ;
		\draw[densely dotted] (ICM) -- (CCM) ;
		\draw (CCM) -- (DCM) -- (ICM) ;
	}
	
	\step{10}[1]{
		\coordinate(N)at($(I)!\framepos!(C)$);% Point à \framepos sur [IC]
		\coordinate(IN)at($(N)-(I)$) ;
		\coordinate(IIN)at($(I)+(IN)$);
		\coordinate(JIN)at($(J)+(IN)$);
		\coordinate(AIN)at($(A)+(IN)$);
		\coordinate(T)at(barycentric cs:I=1,J=1,A=1);
		\coordinate(T')at(barycentric cs:IIN=1,JIN=1,AIN=1);
		\fill[bleu] (IIN) -- (JIN) -- (AIN) -- cycle ;
		\draw[densely dotted] (IIN) -- (JIN) ;
		\draw (JIN) -- (AIN) -- (IIN) ;
		\draw[->](T)--(T') ;
	}
	
	\step{1}{
		\fill[bleu] (IIN) -- (JIN) -- (AIN) -- cycle ;
		\draw[densely dotted] (IIN) -- (JIN) ;
		\draw (JIN) -- (AIN) -- (IIN) ;
	}
	
	\step{10}[1]{
		\coordinate(P)at($(H)!\framepos!(J)$);% Point à \framepos sur [HJ]
		\coordinate(HP)at($(P)-(H)$) ;
		\coordinate(EHP)at($(E)+(HP)$);
		\coordinate(FHP)at($(F)+(HP)$);
		\coordinate(HHP)at($(H)+(HP)$);
		\coordinate(U)at(barycentric cs:E=1,F=1,H=1);
		\coordinate(U')at(barycentric cs:EHP=1,FHP=1,HHP=1);
		\fill[rouge] (EHP) -- (FHP) -- (HHP) -- cycle ;
		\draw[densely dotted] (EHP) -- (HHP) ;
		\draw (EHP) -- (FHP) -- (HHP) ;
		\draw[->](U)--(U') ;
	}
	
	\step{1}{
		\fill[rouge] (EHP) -- (FHP) -- (HHP) -- cycle ;
		\draw[densely dotted] (EHP) -- (HHP) ;
		\draw (EHP) -- (FHP) -- (HHP) ;
	}
	\step{1}{
		\draw[line width=2pt] (C) -- (I) -- (EHP) -- (E) -- cycle ;
		\draw[<->] ($(I)!0.3/\cLen!(EHP)$) --node[fill=blue!15]{$c$} ($(C)!0.3/\cLen!(E)$) ;
		\path (EHP) -- node{${\color{blue}a^2} + {\color{red}b^2} = c^2$} (C) ;
	}
}
\end{center}


\end{document}
