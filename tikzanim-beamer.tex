\documentclass{beamer}

\usepackage{tikzanim}
\usetikzlibrary{calc,intersections,tikzmark,decorations.pathreplacing}

\gdef\aLen{5}
\gdef\bLen{2}
\pgfmathsetmacro{\cLen}{sqrt(\aLen^2+\bLen^2)}
\tikzset{bleu/.style={fill=blue!30,fill opacity=0.5},rouge/.style={fill=red!30,fill opacity=0.5}} 


\begin{document}
	
\begin{frame}{Un titre}
	\begin{itemize}
		\item Une animation :
		
		\tikzanim<2>[poster=0,controls]{5}[>=latex,scale=0.6,every node/.style={rounded corners=2pt,inner sep=1pt}]{
			\useasboundingbox(-0.75,-\bLen-0.75)rectangle(\aLen+\bLen+0.75,\aLen+0.75) ;
		}{
			\step[1]{0}[1]{
				
				\coordinate(A)at(0,0) ;
				\coordinate(B)at(\aLen,0) ;
				\coordinate(C)at(\aLen,\aLen) ;
				\coordinate(D)at(0,\aLen) ;
				
				\coordinate(E)at(\aLen+\bLen,0) ;
				\coordinate(F)at(\aLen+\bLen,\bLen) ;
				\coordinate(G)at(\aLen,\bLen) ;

				\path[name path=EC](E)--(C) ;
				\path[name path=FG](F)--(G) ;
				\path[name intersections={of=EC and FG,by=H}] ;

				\coordinate(I)at(0,\aLen-\bLen) ;
				\coordinate(J)at($(A)+(H)-(G)$);
				\fill[bleu] (C) -- (D) -- (I) -- cycle ;
				
				\path(A) --node{$a^2$} (C) ;
				\path(E) --node{$b^2$} (G) ;
				
			}
			\step{0}[4.0]{
				\fill[bleu] (I) -- (A) -- (J) -- cycle ;
			}
			\step{0}[5.0]{
				\fill[rouge] (E) -- (F) -- (H) -- cycle ;
			}
			\step{1}{
				\draw[blue,line width=2pt] (A) -- (B) -- (C) -- (D) --cycle ;
				\draw[red,line width=2pt] (B) -- (E) -- (F) -- (G) --cycle ;
				
				\fill[bleu] (I) -- (J) -- (B) -- (C) -- cycle ;
				\draw (J) -- (B) -- (C) ; \draw[densely dotted] (C) -- (I) -- (J) ;
					
				\fill[rouge] (B) -- (E) -- (H) -- (G) -- cycle ;
				\draw (H) -- (G) -- (B) -- (E) ; \draw[densely dotted] (E) -- (H) ;
				
				\draw[<->]($(I)-(0.3,0)$)--node[fill=white]{$b$} ($(D)-(0.3,0)$) ;
				\draw[<->]($(D)+(0,0.3)$)--node[fill=white]{$a$} ($(C)+(0,0.3)$) ;
				\draw[<->]($(E)+(0.3,0)$)--node[fill=white]{$b$} ($(F)+(0.3,0)$) ;
				
			}

			\step[5]{10}[1]{
				\coordinate(M)at($(C)!\framepos!(E)$);% Point à \framepos sur [CE]
				\coordinate(CM)at($(M)-(C)$) ;
				\coordinate(ICM)at($(I)+(CM)$);
				\coordinate(CCM)at($(C)+(CM)$);
				\coordinate(DCM)at($(D)+(CM)$);
				\coordinate(S)at(barycentric cs:I=1,C=1,D=1);
				\coordinate(S')at(barycentric cs:ICM=1,CCM=1,DCM=1);
				\fill[bleu] (ICM) -- (CCM) -- (DCM) -- cycle ;
				\draw[densely dotted] (ICM) -- (CCM) ;
				\draw (CCM) -- (DCM) -- (ICM) ;
				\draw[->](S)--(S') ;
			}
			
			\step{1}{
				\fill[bleu] (ICM) -- (CCM) -- (DCM) -- cycle ;
				\draw[densely dotted] (ICM) -- (CCM) ;
				\draw (CCM) -- (DCM) -- (ICM) ;
			}
			
			\step{10}[1]{
				\coordinate(N)at($(I)!\framepos!(C)$);% Point à \framepos sur [IC]
				\coordinate(IN)at($(N)-(I)$) ;
				\coordinate(IIN)at($(I)+(IN)$);
				\coordinate(JIN)at($(J)+(IN)$);
				\coordinate(AIN)at($(A)+(IN)$);
				\coordinate(T)at(barycentric cs:I=1,J=1,A=1);
				\coordinate(T')at(barycentric cs:IIN=1,JIN=1,AIN=1);
				\fill[bleu] (IIN) -- (JIN) -- (AIN) -- cycle ;
				\draw[densely dotted] (IIN) -- (JIN) ;
				\draw (JIN) -- (AIN) -- (IIN) ;
				\draw[->](T)--(T') ;
			}
			
			\step{1}{
				\fill[bleu] (IIN) -- (JIN) -- (AIN) -- cycle ;
				\draw[densely dotted] (IIN) -- (JIN) ;
				\draw (JIN) -- (AIN) -- (IIN) ;
			}
			
			\step{10}[1]{
				\coordinate(P)at($(H)!\framepos!(J)$);% Point à \framepos sur [HJ]
				\coordinate(HP)at($(P)-(H)$) ;
				\coordinate(EHP)at($(E)+(HP)$);
				\coordinate(FHP)at($(F)+(HP)$);
				\coordinate(HHP)at($(H)+(HP)$);
				\coordinate(U)at(barycentric cs:E=1,F=1,H=1);
				\coordinate(U')at(barycentric cs:EHP=1,FHP=1,HHP=1);
				\fill[rouge] (EHP) -- (FHP) -- (HHP) -- cycle ;
				\draw[densely dotted] (EHP) -- (HHP) ;
				\draw (EHP) -- (FHP) -- (HHP) ;
				\draw[->](U)--(U') ;
			}
			
			\step{1}{
				\fill[rouge] (EHP) -- (FHP) -- (HHP) -- cycle ;
				\draw[densely dotted] (EHP) -- (HHP) ;
				\draw (EHP) -- (FHP) -- (HHP) ;
			}
			\step{1}{
				\draw[line width=2pt] (C) -- (I) -- (EHP) -- (E) -- cycle ;
				\draw[<->] ($(I)!0.3/\cLen!(EHP)$) --node[fill=blue!15]{$c$} ($(C)!0.3/\cLen!(E)$) ;
				\path (EHP) -- node{${\color{blue}a^2} + {\color{red}b^2} = c^2$} (C) ;
			}
		}
		
		Après l'animation.
		
		\item<3-> Un troisième {\ttfamily overlay}.
	\end{itemize}
\end{frame}		

\end{document}
