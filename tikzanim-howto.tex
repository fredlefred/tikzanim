\documentclass[a4paper,12pt]{article}

\usepackage[T1]{fontenc}
\usepackage{amsmath}
\usepackage{multicol}
\usepackage{enumitem}
\usepackage{tikzanim}
\usepackage[francais]{babel}
\usetikzlibrary{calc,intersections,tikzmark,decorations.pathreplacing}

\usepackage{hyperref}

\usepackage{listings}
\lstset{
	basicstyle=\ttfamily\footnotesize,
	columns=fullflexible,
	language={[LaTeX]TeX},
	tabsize=2,
	xleftmargin=2em,
	backgroundcolor=\color{black!10},
	numbers=left,
	numberstyle=\color{black!50},
	commentstyle=\color{gray},
	keywordstyle=\color{teal!75!black},
	morekeywords={draw,coordinate,useasboundingbox},
	identifierstyle=\color{blue!75!black},
	emph={tikzanim,step,allframes,framepos},
	emphstyle=\color{purple},
	commentstyle=\color{green!75!black},
}

\RequirePackage{vmargin,fancyhdr}

\setmarginsrb{1cm}{1cm}{1cm}{1cm}{0cm}{0cm}{0cm}{1cm}

\usetikzlibrary{calc}% a must to calculate animate steps

\tikzset{construct line/.style={densely dashed,black!25}}

\def\keywords{make easily animation with animate package and tikz}
\hypersetup{
  pdftitle={Le Package tikzanim},
  pdfsubject={Documentation},
  pdfauthor={Frédéric Bonnaud},
  pdfkeywords={\keywords},
    pdfborder = {0 0 0},
    colorlinks,
    citecolor=red,
    filecolor=Darkgreen,
    linkcolor=blue,
    urlcolor=cyan!50!black!90
}
\author{Frédéric Bonnaud}%
\title{\TikzAnimate\\ Créer une animation avec \Tikz}
\date{15 juin 2020}

\newcommand\Tikz{{\ttfamily tikz}}
\newcommand\TikzAnimate{{\ttfamily tikzanim}}
\newcommand\Animate{{\ttfamily animate}}

\setlength{\parindent}{0mm}
\begin{document}

\maketitle

\tableofcontents

\section{Introduction}

\TikzAnimate\ permet de réaliser simplement des animations en \LaTeX à l'aide de \Tikz\ et \Animate\ en décrivant les différentes étapes de l'animation par un code \Tikz\ standard.

Une animation est constituée d'un certaine nombre d'étapes. Chaque étape dessine des transparents qui sont combinés les uns aux autres pour créer les images du morceau d'animation qu'elle constitue.


\section{Les macros offertes par \TikzAnimate}

\subsection{\ttfamily\textbackslash tikzanim}

	Cette macro installe tout ce qu'il faut pour mettre en place l'animation : un environnement {\ttfamily animateinline} (pour créer l'animation), la macro {\ttfamily\textbackslash multiframe} (pour créer les différents images) et un environnement {\ttfamily\textbackslash tikzpicture} (pour dessiner les différentes images).
	
	\textbf{Syntaxe~:} 
	
	\quad {\ttfamily \textbackslash tikzanim [options animateinline] \{rafraîchissement\} [options tikz] \{étapes\}}
	
	\begin{itemize}
		\item {\ttfamily [options animateinline]} : les options qui seront passées à l'environnement {\ttfamily animateinline}.
		
		Valeur par défaut : {\ttfamily poster=last,controls}
		\item {\ttfamily \{rafraîchissement\}} : le nombre d'images par seconde au début de l'animation. Il peut être modifié à chaque étape.
		\item {\ttfamily [options tikz]} : les options qui seront passée à l'environnement {\ttfamily tikzpicture}.
		
		Valeur par défaut : pas d'options
		\item {\ttfamily \{étapes\}} : la suite des différentes étapes de l'animation. 
	\end{itemize}
	
\subsection{\ttfamily\textbackslash step}

	Cette macro définit une nouvelle étape de l'animation. \Animate\ définit une pile de transparents. Chaque image de l'animation est générée par l'affichage 
	de toutes la pile de transparents à chaque instant. Le rôle de {\ttfamily\textbackslash step} est de gérer la création (avec \Tikz) et la gestion cette pile
	(en générant un fichier timeline) pour créer les différentes images qui permettrons créer l'animation.
		
	Au début de l'animation, la pile est vide. Un appel à {\ttfamily\textbackslash step} va :
	
	\begin{enumerate}[label=(\arabic*)]
		\item Générer la prermière image de cette étape et la déposer sur la pile pour une durée de un image.
		\item \tikzmark{boucle}Enlever la dernière image de la pile.
		\item Générer l'image suivante et la déposer sur la pile pour une durée de un image.
		\item \tikzmark{test}Si ce n'est pas la fin de l'étape, on boucle vers (2).
		\item Si c'est la fin de l'étape, on laisse la dernière image pour une durée définie.
		
	\end{enumerate}
	\begin{tikzpicture}[remember picture,overlay]
		\coordinate(T)at(-10mm,1mm);
		\draw[-latex,thick] ($(pic cs:test)+(T)$) to[bend left] ($(pic cs:boucle)+(T)$) ; 
	\end{tikzpicture}
	
	\textbf{Syntaxe :}
	
	\quad {\ttfamily\textbackslash step[*] [rafraîchissement] \{images\} [durée] \{initialisation\} \{dessin\} }
	
	\begin{itemize}
		\item {\ttfamily *} : la version étoilée, vide la pile de transparents au début de l'étape.
		\item {\ttfamily[rafraîchissement]} : définit le nouveau nombre d'images par seconde.
		
		Valeur par défaut : 0, c'est à dire le dernier nombre d'images par seconde définit par {\ttfamily\textbackslash tikzanim} ou {\ttfamily\textbackslash step}.
		\item {\ttfamily\{images\}} : définit le nombre de images que doit générer {\ttfamily\textbackslash step}. Si ce nombre est 0. L'image est créée, mais aucune image n'est créée
		pour elle. L'image peut tout de même avoir une durée. Cela permet de dessiner dans des étapes différentes des objets qui auront des durées d'affichage différentes.
		
		\item {\ttfamily[durée]} : définit le temps pendant lequel la derniére image doit durer (en nombre d'images).
		
		Valeur par défaut : 0, c'est à dire jusqu'à la fin de l'animation.
		\item {\ttfamily\{initialisation\}} : définit le code \Tikz\ d'initialisation. Comme l'animation est générée par étape, pour pouvoir réutiliser des éléments 
		des étapes précédentes aux étapes suivantes, il faut que ces éléments soit définis. C'est dans le code d'initialisation qu'il faut le faire.
		
		Pour que le code d'initialisation dépende du transparent en train d'être générer, on peut utiliser :
		\begin{itemize}
			\item {\ttfamily\textbackslash framepos} qui vaut 0 au  début de l'étape et varie de façon linéaire jusqu'à 1 à la fin de l'étape. Cette macro est gérée par {\ttfamily\textbackslash step}.
			\item {\ttfamily\textbackslash iframe} qui est égale au numéro du transparent en cours de génération. Cette macro est gérée par {\ttfamily\textbackslash multiframe} du package \Animate.
		\end{itemize}
		
		\emph{Remarque~: pour fixer une taille identique pour toutes les images, le code d'initialisation de la première étape devrait toujours contenir un appel à 
		{\ttfamily\textbackslash useasboundingbox} }

		\item {\ttfamily\{dessin\}} : définit le code \Tikz\ chargé de dessiner les différents transparents. Ce code peut utiliser tout ce qui se trouve dans
		les codes d'initialisation des étapes précédentes, ainsi que {\ttfamily\textbackslash framepos} et {\ttfamily\textbackslash iframe}.
	\end{itemize}
	
	\begin{multicols}{2}
	{Le code suivant~:}
	\begin{lstlisting}
% \usepackage{tikzanim}
\tikzanim{10}{
	% cette étape se prolongera sur 15 images
	% c'est à dire la moitié de l'étape suivante
	\step{30}[15]{
		% il faut initialiser A et B ici, car
		% la `boundingbox' doit être utilisée
		% à l'étape suivante
		\coordinate(A)at(0,0) ;
		\coordinate(B)at(5,5) ;
		\useasboundingbox(A)rectangle(B) ;
	}{
		\draw(A)--($(A)!\framepos!(B)$) ;
	}
	% la seconde étape sera 2 fois plus rapide
	\step[20]{30}{
		% pas d'initialisation, C et D ne
		% sont pas utilisés par la suite
	}{
		\coordinate(C)at(0,5) ;
		\coordinate(D)at(5,0) ;
		\draw(C)--($(C)!\framepos!(D)$) ;
	}
}
	\end{lstlisting}
	
	{Crée l'animation~:}
	
	\bigskip
	
	\centering
	% \usepackage{tikzanim}
	\tikzanim{10}{
		\step{30}[15]{
			\coordinate(A)at(0,0) ;
			\coordinate(B)at(5,2.5) ;
			\useasboundingbox(A)rectangle(B) ;
		}{
			\draw(A)--($(A)!\framepos!(B)$) ;
		}
		\step[20]{30}{
			% pas d'initialisation, C et D ne
			% sont pas utilisés par la suite
		}{
			\coordinate(C)at(0,2.5) ;
			\coordinate(D)at(5,0) ;
			\draw(C)--($(C)!\framepos!(D)$) ;
		}
	}
	
	\end{multicols}
	
	\textbf{Schéma de fonctionnement :}
	
	\medskip
	
\newcounter{dy} 
\newcommand{\transparentlist}[2]{
	\stepcounter{dy}	
	\foreach\f in {#1}{
		\pgfmathsetmacro{\c}{\f*5}
		\path[draw=black,fill=teal!\c!purple!75!white,fill opacity=0.75] (\f*0.1,-\thedy*0.8)node(a){} -- ++(0,0.6) -- ++(0.3,0.3)node(b){} -- ++(0,-0.6) -- cycle ;
		\path(a)--node[midway,font=\tiny,shape=rectangle]{\f} (b) ;
	}
	\node[anchor=west,shape=rectangle] at(2.5,0.4-\thedy*0.8) {#2#1} ;
}

\newcommand{\acc}[4][0pt]{
		\draw[decorate,decoration={brace,amplitude=10pt},thick,xshift=#1] (12,-0.8*#2-0.05) --node[right,xshift=10pt,text width=7cm,font=\ttfamily,shape=rectangle]{#4} (12,-0.8*#3+0.05);
}

\newcommand{\entoure}[4][red,thick,densely dashed]{
	\draw[#1] (#2*0.1-0.05,-#4*0.8-0.125) -- ++(0,{(#4-#3+1)*0.8-0.05}) -- ++(0.4,0.4) -- ++ (0,{-(#4-#3+1)*0.8+0.05}) -- cycle ;
}

\newcommand{\limite}[2][teal!50,thick,densely dotted]{
	\draw[#1](-0.2,-0.8*#2) -- (11.9,-0.8*#2) ;
}


\begin{tikzpicture}[shape=coordinate]

	\entoure{0}{1}{18}

	%\def\label{Liste de transparents : }
	%\transparentlist{0,...,20}{\label}
	\def\label{Image \thedy, constituée des transparents : }
	
	
	\transparentlist{0,1}{\label}
	\transparentlist{0,2}{\label}
	\transparentlist{0,3}{\label}
	\transparentlist{0,4}{\label}
	\transparentlist{0,5}{\label}
	
	\entoure{6}{6}{9}
	
	\transparentlist{0,6}{\label}
	
	\acc{0}{6}{ 
		\textbackslash step\{0\}{\color{red}[18]}\{...\}\\ 
		{\sffamily ajoute 1 transparent qui dure 18 images, sans ajouter d'image}\\
		\textbackslash step\{6\}{\color{red}[4]}\{...\}\\
		{\sffamily ajoute 6 transparents et 6 images, le dernier dure 4 images}
	}
	
	\limite{6}
	
	\transparentlist{0,6,7}{\label}
	\transparentlist{0,6,8}{\label}
	\transparentlist{0,6,9}{\label}
	\transparentlist{0,10}{\label}
	\transparentlist{0,11}{\label}

	\acc{6}{11}{ 
		\textbackslash step\{5\}\{...\} \\
		{\sffamily ajoute 5 transparents et 5 images}
	}
	
	\limite{11}
	
	\transparentlist{0,11,12}{\label}
	\transparentlist{0,11,13}{\label}
	\transparentlist{0,11,14}{\label}
	\transparentlist{0,11,15}{\label}
	\entoure{16}{16}{18}
	\transparentlist{0,11,16}{\label}
	
	\acc{11}{16}{
		\textbackslash step\{5\}{\color{red}[3]}\{...\} \\
		{\sffamily ajoute 5 transparents et 5 images, le dernier dure 3 images}
	}
	
	\limite{16}
	
	\transparentlist{0,11,16,17}{\label}
	\transparentlist{0,11,16,18}{\label}
	\transparentlist{11,19}{\label}
	\transparentlist{11,20}{\label}
	
	\acc{16}{20}{
		\textbackslash step\{4\}\{...\} \\
		{\sffamily ajoute 4 transparents et 4 images}
	}

\end{tikzpicture}
	

\subsection{\ttfamily\textbackslash allframes}

Cette macro permet de dessiner sur tous les transparents.

\textbf{Syntaxe~:}

\quad {\ttfamily\textbackslash allframes \{dessin\}}

	\begin{itemize}
		\item {\ttfamily\{dessin\}} : définit le code \Tikz\ chargé de dessiner sur tous les transparents. Ce code peut utiliser tout ce qui se trouve dans
		les codes d'initialisation des étapes précédentes, ainsi que {\ttfamily\textbackslash framepos} et {\ttfamily\textbackslash iframe}.
	\end{itemize}
	
\section{\ttfamily timeline}

\Animate\ utilise un fichier {\ttfamily timeline} pour gérer l'affichage des transparents. \TikzAnimate\ crée un fichier {\ttfamily timeline} par animation. Ils sont nommés : {\ttfamily \textbackslash jobname.tzc\#.tln}. Ils peuvent être utiliser à des fins de débuggage de l'animation.
	
\section{Un exemple complet}

\begin{lstlisting}[name=exemplecomplet]
\gdef\aLen{5}
\gdef\bLen{2}
\pgfmathsetmacro{\cLen}{sqrt(\aLen^2+\bLen^2)}
\tikzset{bleu/.style={fill=blue!30,fill opacity=0.5},rouge/.style={fill=red!30,fill opacity=0.5}} 
\end{lstlisting}

Quelques initialisations.


\begin{lstlisting}[name=exemplecomplet]

\tikzanim[poster=0,controls]{1}{
\end{lstlisting}
On crée une animation à une image par seconde (au départ).

Puis, on crée les points servants à la construction, en même temps qu'un permier transparent qui ne sera affiché qu'à la prochaine image.
\begin{lstlisting}[name=exemplecomplet]
	\step{0}[1]{
		% initialisation 
		\useasboundingbox(0,-\bLen)rectangle(\aLen+\bLen,\aLen) ;
		
		\coordinate(A)at(0,0) ;
		\coordinate(B)at(\aLen,0) ;
		\coordinate(C)at(\aLen,\aLen) ;
		\coordinate(D)at(0,\aLen) ;

		\coordinate(E)at(\aLen+\bLen,0) ;
		\coordinate(F)at(\aLen+\bLen,\bLen) ;
		\coordinate(G)at(\aLen,\bLen) ;

		\path[name path=EC](E)--(C) ;
		\path[name path=FG](F)--(G) ;
		\path[name intersections={of=EC and FG,by=H}] ;

		\coordinate(I)at(0,\aLen-\bLen) ;
		\coordinate(J)at($(A)+(H)-(G)$);
	}{
		% premier morceau
		\fill[bleu] (C) -- (D) -- (I) -- cycle ;
	}
\end{lstlisting}

On crée un nouveau transparent qui ne sera affiché qu'avec la prochaine image.

\begin{lstlisting}[name=exemplecomplet]
	\step{0}[12]{}{
		% deuxième morceau
		\fill[bleu] (I) -- (A) -- (J) -- cycle ;
	}
\end{lstlisting}

On crée encore un nouveau transparent qui ne sera affiché qu'avec la prochaine image.

\begin{lstlisting}[name=exemplecomplet]
	\step{0}[23]{}{
		% troisième morceau
		\fill[rouge] (E) -- (F) -- (H) -- cycle ;
	}
\end{lstlisting}

On crée un seul transparent, qui contient ce qui va rester affiché jusqu'à la fin (et qui donc ne bougera pas) et les 3 transparents précédants qui disparaîtront lorsque le mouvement de ce qu'ils 
contiennent commencera.

\begin{lstlisting}[name=exemplecomplet]
	\step{1}{
	}{
		% figure de départ sans les morceaux
		\draw (A) -- (B) -- (C) -- (D) --cycle ;
		\draw (B) -- (E) -- (F) -- (G) --cycle ;
		
		\fill[bleu] (I) -- (J) -- (B) -- (C) -- cycle ;
		\draw (J) -- (B) -- (C) ; \draw[densely dotted] (C) -- (I) -- (J) ;
			
		\fill[rouge] (B) -- (E) -- (H) -- (G) -- cycle ;
		\draw (H) -- (G) -- (B) -- (E) ; \draw[densely dotted] (E) -- (H) ;
	}
\end{lstlisting}
Le mouvement commence, on fixe le nombre d'image par seconde à 5, et on génére 10 images, cette étape va donc prendre 2 secondes. Comme sa durée est de 1 image, cela signifie 
que le dernier transparent ne sera pas affiché sur les images qui suivent cette étape.
\begin{lstlisting}[name=exemplecomplet]
	% déplacement du premier triangle
	\step[5]{10}[1]{
		\coordinate(M)at($(C)!\framepos!(E)$);% Pt à #1 sur [CE]
		\coordinate(CM)at($(M)-(C)$) ;
		\coordinate(ICM)at($(I)+(CM)$);
		\coordinate(CCM)at($(C)+(CM)$);
		\coordinate(DCM)at($(D)+(CM)$);
	}{
		% on affiche une flèche en plus du triangle pour bien voir d'où vient le triangle.
		\coordinate(S)at(barycentric cs:I=1,C=1,D=1);
		\coordinate(S')at(barycentric cs:ICM=1,CCM=1,DCM=1);
		\fill[bleu] (ICM) -- (CCM) -- (DCM) -- cycle ;
		\draw[densely dotted] (ICM) -- (CCM) ;
		\draw (CCM) -- (DCM) -- (ICM) ;
		\draw[-latex](S)--(S') ;
	}
\end{lstlisting}

Comme dernier transparent ne dure qu'une image, le triangle final ne reste pas dessiné. On le dessine à nouveau.

\begin{lstlisting}[name=exemplecomplet]
	% position finale
	\step{1}{}{
		\fill[bleu] (ICM) -- (CCM) -- (DCM) -- cycle ;
		\draw[densely dotted] (ICM) -- (CCM) ;
		\draw (CCM) -- (DCM) -- (ICM) ;
	}
\end{lstlisting}

On procède de même pour les 2 autres triangles.

\begin{lstlisting}[name=exemplecomplet]
	% déplacement du deuxième triangle
	\step{10}[1]{
		\coordinate(N)at($(I)!\framepos!(C)$);% Pt à #1 sur [IC]
		\coordinate(IN)at($(N)-(I)$) ;
		\coordinate(IIN)at($(I)+(IN)$);
		\coordinate(JIN)at($(J)+(IN)$);
		\coordinate(AIN)at($(A)+(IN)$);
	}{
		\coordinate(T)at(barycentric cs:I=1,J=1,A=1);
		\coordinate(T')at(barycentric cs:IIN=1,JIN=1,AIN=1);
		\fill[bleu] (IIN) -- (JIN) -- (AIN) -- cycle ;
		\draw[densely dotted] (IIN) -- (JIN) ;
		\draw (JIN) -- (AIN) -- (IIN) ;
		\draw[-latex](T)--(T') ;
	}
	% position finale
	\step{1}{}{
		\fill[bleu] (IIN) -- (JIN) -- (AIN) -- cycle ;
		\draw[densely dotted] (IIN) -- (JIN) ;
		\draw (JIN) -- (AIN) -- (IIN) ;
	}
	% déplacement du troisième triangle
	\step{10}[1]{
		\coordinate(P)at($(H)!\framepos!(J)$);% Pt à #1 sur [HJ]
		\coordinate(HP)at($(P)-(H)$) ;
		\coordinate(EHP)at($(E)+(HP)$);
		\coordinate(FHP)at($(F)+(HP)$);
		\coordinate(HHP)at($(H)+(HP)$);
	}{
		\coordinate(U)at(barycentric cs:E=1,F=1,H=1);
		\coordinate(U')at(barycentric cs:EHP=1,FHP=1,HHP=1);
		\fill[rouge] (EHP) -- (FHP) -- (HHP) -- cycle ;
		\draw[densely dotted] (EHP) -- (HHP) ;
		\draw (EHP) -- (FHP) -- (HHP) ;
		\draw[-latex](U)--(U') ;
	}
	% position finale
	\step{1}{}{
		\fill[rouge] (EHP) -- (FHP) -- (HHP) -- cycle ;
		\draw[densely dotted] (EHP) -- (HHP) ;
		\draw (EHP) -- (FHP) -- (HHP) ;
	}
}
\end{lstlisting}

\textbf{L'animation :}

Une démonstration visuelle du théorème de Pythagore.

\begin{center}

\gdef\aLen{5}
\gdef\bLen{2}
\pgfmathsetmacro{\cLen}{sqrt(\aLen^2+\bLen^2)}
\tikzset{bleu/.style={fill=blue!30,fill opacity=0.5},rouge/.style={fill=red!30,fill opacity=0.5}} 

\tikzanim[poster=0,controls]{5}{
	\step[1]{0}[1]{
		\useasboundingbox(0,-\bLen)rectangle(\aLen+\bLen,\aLen) ;
		
		\coordinate(A)at(0,0) ;
		\coordinate(B)at(\aLen,0) ;
		\coordinate(C)at(\aLen,\aLen) ;
		\coordinate(D)at(0,\aLen) ;
		
		\coordinate(E)at(\aLen+\bLen,0) ;
		\coordinate(F)at(\aLen+\bLen,\bLen) ;
		\coordinate(G)at(\aLen,\bLen) ;

		\path[name path=EC](E)--(C) ;
		\path[name path=FG](F)--(G) ;
		\path[name intersections={of=EC and FG,by=H}] ;

		\coordinate(I)at(0,\aLen-\bLen) ;
		\coordinate(J)at($(A)+(H)-(G)$);
	}{
		\fill[bleu] (C) -- (D) -- (I) -- cycle ;
	}
	\step{0}[12]{}{
		\fill[bleu] (I) -- (A) -- (J) -- cycle ;
	}
	\step{0}[23]{}{
		\fill[rouge] (E) -- (F) -- (H) -- cycle ;
	}
	\step{1}{
	}{
		\draw (A) -- (B) -- (C) -- (D) --cycle ;
		\draw (B) -- (E) -- (F) -- (G) --cycle ;
		
		\fill[bleu] (I) -- (J) -- (B) -- (C) -- cycle ;
		\draw (J) -- (B) -- (C) ; \draw[densely dotted] (C) -- (I) -- (J) ;
			
		\fill[rouge] (B) -- (E) -- (H) -- (G) -- cycle ;
		\draw (H) -- (G) -- (B) -- (E) ; \draw[densely dotted] (E) -- (H) ;
	}

	\step[5]{10}[1]{
		\coordinate(M)at($(C)!\framepos!(E)$);% Pt à #1 sur [CE]
		\coordinate(CM)at($(M)-(C)$) ;
		\coordinate(ICM)at($(I)+(CM)$);
		\coordinate(CCM)at($(C)+(CM)$);
		\coordinate(DCM)at($(D)+(CM)$);
	}{
		\coordinate(S)at(barycentric cs:I=1,C=1,D=1);
		\coordinate(S')at(barycentric cs:ICM=1,CCM=1,DCM=1);
		\fill[bleu] (ICM) -- (CCM) -- (DCM) -- cycle ;
		\draw[densely dotted] (ICM) -- (CCM) ;
		\draw (CCM) -- (DCM) -- (ICM) ;
		\draw[-latex](S)--(S') ;
	}
	
	\step{1}{}{
		\fill[bleu] (ICM) -- (CCM) -- (DCM) -- cycle ;
		\draw[densely dotted] (ICM) -- (CCM) ;
		\draw (CCM) -- (DCM) -- (ICM) ;
	}
	
	\step{10}[1]{
		\coordinate(N)at($(I)!\framepos!(C)$);% Pt à #1 sur [IC]
		\coordinate(IN)at($(N)-(I)$) ;
		\coordinate(IIN)at($(I)+(IN)$);
		\coordinate(JIN)at($(J)+(IN)$);
		\coordinate(AIN)at($(A)+(IN)$);
	}{
		\coordinate(T)at(barycentric cs:I=1,J=1,A=1);
		\coordinate(T')at(barycentric cs:IIN=1,JIN=1,AIN=1);
		\fill[bleu] (IIN) -- (JIN) -- (AIN) -- cycle ;
		\draw[densely dotted] (IIN) -- (JIN) ;
		\draw (JIN) -- (AIN) -- (IIN) ;
		\draw[-latex](T)--(T') ;
	}
	
	\step{1}{}{
		\fill[bleu] (IIN) -- (JIN) -- (AIN) -- cycle ;
		\draw[densely dotted] (IIN) -- (JIN) ;
		\draw (JIN) -- (AIN) -- (IIN) ;
	}
	
	\step{10}[1]{
		\coordinate(P)at($(H)!\framepos!(J)$);% Pt à #1 sur [HJ]
		\coordinate(HP)at($(P)-(H)$) ;
		\coordinate(EHP)at($(E)+(HP)$);
		\coordinate(FHP)at($(F)+(HP)$);
		\coordinate(HHP)at($(H)+(HP)$);
	}{
		\coordinate(U)at(barycentric cs:E=1,F=1,H=1);
		\coordinate(U')at(barycentric cs:EHP=1,FHP=1,HHP=1);
		\fill[rouge] (EHP) -- (FHP) -- (HHP) -- cycle ;
		\draw[densely dotted] (EHP) -- (HHP) ;
		\draw (EHP) -- (FHP) -- (HHP) ;
		\draw[-latex](U)--(U') ;
	}
	
	\step{1}{}{
		\fill[rouge] (EHP) -- (FHP) -- (HHP) -- cycle ;
		\draw[densely dotted] (EHP) -- (HHP) ;
		\draw (EHP) -- (FHP) -- (HHP) ;
	}
}

\end{center}

\end{document}
